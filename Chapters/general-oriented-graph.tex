\chapter{General Oriented Graph}\label{chap: general oriented graph}

The properties of semi-complete digraphs basically inherit
the properties of tournaments,
and most of the proofs are almost the same.

We then move on to another family of graphs that are less
similar to tournaments: oriented graphs.
We will investigate general properties of oriented graph
in this chapter.
In later \cref{chap: quasi-transitive}, we will focus on
specific families of oriented graph

\begin{lemma}\label{the: no (4 4) oriented graph}
  There does not exist a \((4,4)\) general oriented graph.
\end{lemma}

\begin{lemma}\label{the: no (2 2) and (3 2) oriented graph}
  There does not exist a \((2, 2)\) oriented graph
  and a \((3, 2)\) oriented graph
\end{lemma}

The previous lemmas can be proved by
listing all the oriented graph with 2, 3, and 4 vertices.
We will not show the proofs in this thesis,
because the proofs are too long and uninteresting.

\begin{lemma}\label{the: (n 2) oriented graph}
  There exists \((n, 2)\) oriented graph for \(n \geq 4\)
\end{lemma}

\begin{proof}
  \begin{figure}
    \centering
    \tikz\graph[tree layout, grow=down, math nodes, sibling distance=1cm,level sep=1cm] {
        a -> b -> {c_1, c_2, "\ldots", c_{n-3}, c_{n-2}};  %chktex 18
        c_1 -> a;
    };
    \caption{only \(a\) and \(b\) are kings for \(n \geq 4\)}
    \label{fig: (n 2) oriented graph}  %chktex 24
  \end{figure}
  We can see in \cref{fig: (n 2) oriented graph} that
  \(c_1\) cannot dominate \(c_2\) in 2 steps.
  and other \(c_i\) cannot dominate any other vertex,
  because there is no edge going out of them.

  \(a\) is a king because \(a \to b \to c_k\),
  where \(k\) is a integer and \(k \leq n - 2\),
  therefore \(a\) beats \(b\) by one step,
  and beats all the \(c_k\) by 2 steps.
  \(b\) beats every \(c_k\) by 2 steps,
  where \(k\) is a positive integer and \(k \leq n - 2\),
  and \(b\) beats \(a\) by 2 steps: \(b \to c_1 \to a\).
\end{proof}

\begin{theorem}\label{the: (n k) oriented graph}
  There exists an \((n, k)\) oriented graph for all the \(n \geq k \geq 1\),
  with the exception of \((2, 2)\), \((2, 3)\) and \((4, 4)\) oriented graph.
\end{theorem}

\begin{proof}
  \cref{the: (n k) tournament exists} shows that
  there exists \((n, k)\) tournament for all \(n \geq k \geq 1\)
  with the exception of \((n, 2)\), and \((4, 4)\).

  Because tournaments are also oriented graphs and
  by \cref{the: (n 2) oriented graph},
  \cref{the: no (2 2) and (3 2) oriented graph}, \cref{the: no (4 4) oriented graph},
  we can conclude that the theorem is correct.
\end{proof}

We generalize the result from~\cite{maurer_king_1980}
on tournaments to oriented graphs
and show that there are only finite number of
\((n, k)\) oriented graphs that cannot be constructed.

Following the idea from \cref{chap: semi-complete digraph},
one of the question to ask is that how can we use
ties more ``efficiently''.
The construction method in the proof of
\cref{the: (n 2) oriented graph} is pretty inefficient.

Here we present a better way to construct the graph.

\begin{lemma}\label{the: (n 2) with one tie}
  There exists \((n, 2)\) oriented graph
  with only one tie, for \(n \geq 4\)
\end{lemma}

\begin{proof}
  \begin{figure}
    \centering
    \tikz\graph[tree layout, math nodes, grow=down, sibling distance=2cm,level sep=0.75cm] {
      a -> b -> {c, T_{n-3} [draw, circle]};
      T_{n-3} -> c;
      T_{n-3} -> [bend right] a;
    };
    \caption{the constructive proof for
      \cref{the: (n 2) with one tie}}
    \label{fig: (n 2) with one tie} %chktex 24
  \end{figure}

  See \cref{fig: (n 2) with one tie},
  the \(T_{n - 3}\) is a tournament of \(n - 3\) vertices.
  In this graph, the only tie is between \(a\) and \(c\)
  and the only kings are \(a\) and \(b\).

  \(a\) is a king because, \(a \to b \to c\)
  and \(a \to b \to T_{n - 3}\),
  therefore \(a\) beats \(b\) in 1 step
  and \(a\) beats \(c\) and \(T_{n-3}\) in 2 steps.
  \(b\) is a king because, \(b \to c\)
  and \(b \to T_{n - 3} \to a\)
  (because \(T_{n-3}\) is not empty),
  therefore \(b\) beats \(c\) and \(T_{n - 3}\),
  and \(b\) beats \(a\) in 2 steps.

  \(c\) is not a king, because it has out-degree 0.
  Any vertex \(v\) in \(T_{n-3}\) cannot be a king,
  because the closest path between \(v\) and \(c\)
  is \(v \to a \to b \to c\) which has length 3.
\end{proof}

With \cref{the: (n 2) with one tie},
we can prove the following theorem.

\begin{theorem}\label{the: (n k) oriented graph with one tie}
  There exists an \((n, k)\) oriented graph
  with one tie for all the \(n \geq k \geq 1\),
  with the exception of
  \((2, 2)\), \((2, 3)\) and \((4, 4)\) oriented graph.
\end{theorem}

\begin{proof}
  Almost the same proof as \cref{the: (n k) oriented graph},
  just substitute \cref{the: (n 2) oriented graph} with
   \cref{the: (n 2) with one tie}.
\end{proof}

in other words,
\cref{the: (n k) oriented graph with one tie},
states that for all \((n, k)\) oriented graphs
that can be constructed, can be constructed with only one tie.

%TODO: strong and locally semi-complete
