\section{Graph Condensations}\label{sec: graph condensation}

\begin{definition}
  A \keyword{condensation} is a function \(f: G \to H\),
  where \(G\) and \(H\) are oriented graphs,
  and \(f\) maps \(V(G)\) to \(V(H)\) surjectively,
  such that
  \begin{itemize}
    \item for any two vertices \(a, b \in V(G)\),
      if \(f(a) \to f(b)\) in \(H\)
      then \(a \to b\) in \(G\).
    \item for any two vertices \(a, b \in V(G)\),
      if \(f(a)\) ties \(f(b)\) in \(H\)
      then \(a\) ties \(b\) in \(G\).
  \end{itemize}
  We call \(G\) the \keyword{uncondensed graph},
  and \(H\) the \keyword{condensed graph}.
\end{definition}

Two examples of condensations are identity condensation
and trivial condensation.

Identity condensation \(i\) maps an oriented graph \(G\) to itself,
such that \(i\) maps every vertex to itself,
and \(f(a) \to f(b)\) if and only if \(a \to b\).
The identity condensation preserves all the edges
and ties, and do not change the graph at all.
We can see that this map is a condensation by definition.

Another example is trivial condensation \(t\),
which maps any oriented graph \(G\) into a single point.
Because for any two vertices \(a, b \in G\),
\(t(a) = t(b)\), therefore by definition,
\(t(a)\) does not tie or beat \(t(b)\).
Therefore, this mapping also satisfies
the definition of condensation.

\begin{definition}
  Given oriented graph \(G\) as the uncondensed graph of
  condensation \(f\),
  A \keyword{component} of vertex \(a \in V(G)\)
  is the pre-image of \(f(a)\),
  in other words, the component of \(a\) is
  \(\set{x \in V(G) \mid f(x) = f(a)}\).
\end{definition}

A component of vertex \(a\) is a set of vertices
that are condensed into the same vertex as \(a\).
For example, the component of vertex \(a\)
in oriented graph \(G\) with identity condensation
is a set that only contains \(a\) itself,
and the component of a vertex \(a\) in oriented graph \(G\)
with the trivial condensation is set of all the vertices,
because all vertices in \(G\) are mapped onto a
single vertex in \(H\),
therefore all the vertices in \(G\) are
condensed into the same vertex as \(a\).

\begin{figure}
  \centering
  \begin{subfigure}[b]{0.45\linewidth}
    \centering
    \tikz\graph[simple necklace layout, math nodes, node sep=1cm] {
      a; b;
      a --[dashed] {c, d, e};
      b --[dashed] {c, d, e};
      {c, d, e} -> f;
      {a, b} -> f;
      j -> {c, d, e};
      {a, b} -> j;
      f -> j;
      c -> d -> e --[dashed] c;
      a -> b;
    };
    \caption{an oriented graph \(G\).}
  \end{subfigure}
  \begin{subfigure}[b]{0.45\linewidth}
    \centering
    \tikz\graph[simple necklace layout, math nodes, node sep=1cm] {
      a; b; c; d; e; f; j;
      d -> e --[dashed] c;
      a -> b;
      / [label = left:\(A\), draw, circle] // {a, b};
      / [label = left:\(B\), draw] // {d, e, c};
      / [label = right:\(C\), draw, circle] // {f};
      / [label = right:\(D\), draw, circle] // {j};
    };
    \caption{components of \(G\) in this condensation.}
  \end{subfigure}
  \begin{subfigure}[b]{0.45\linewidth}
    \centering
    \tikz\graph[simple necklace layout, math nodes, node sep=1cm] {
      A [draw, circle, minimum size=1.5cm] --[dashed]
      B [draw, circle, minimum size=1.75cm] ->
      C [draw, circle, minimum size=1cm] ->
      D [draw, circle, minimum size=1cm];
      A -> D -> B
    };
    \caption{find the relationship between components.}
  \end{subfigure}
  \begin{subfigure}[b]{0.45\linewidth}
    \centering
    \tikz\graph[simple necklace layout, math nodes, node sep=1cm] {
      A' --[dashed] B' -> C' -> D';
      A' -> D' -> B';
    };
    \caption{condensed graph \(H\).}
  \end{subfigure}
  \caption{an example of graph condensation.}
  \label{fig: condensation example}  % chktex 24
\end{figure}

In \cref{fig: condensation example},
we show a general example of condensation.
In this example,
the distinct components in \(G\) are \(A\), \(B\), \(C\) and \(D\),
and they are condensed into vertices
\(A'\), \(B'\), \(C'\) and \(D'\) respectively.
The component of \(a\) is \(A\),
the component of \(d\) is \(B\),
and the component of \(c\) is also \(B\),
because both \(c\) and \(d\) are mapped into vertex
\(B'\) in oriented graph \(H\).

We proof several theorems involving condensation and components
to help us with our goal of identifying kings in
quasi-transitive oriented graphs.

\begin{theorem}\label{the: condensation component partition}
  Given any oriented graph \(G\) and condensation \(f\),
  distinct components in \(G\) partition \(V(G)\).
\end{theorem}

\begin{proof}
  Prove distinct component disjoint:
  Assume there exists two vertex \(a\) and \(b\),
  such that the component of \(a\) is \(A\);
  component of \(b\) is \(B\),
  and \(A \neq B\); \(A \cap B \neq \emptyset \).
  Therefore, there exists \(v \in A \cap B\).
  By definition of components,
  \(f(v) = f(a)\), and \(f(v) = f(b)\),
  therefore \(f(b) = f(a)\),
  and \(B
  = \set{x \in V(G) \mid f(x) = f(b)}
  = \set{x \in V(G) \mid f(x) = f(a)}
  = A\).
   \(A = B\), contradiction.

  Prove every vertex is in a component:
  for every vertex \(v \in V(G)\),
  \(v\) is in the component of \(v\) because \(f(v) = f(v)\).
\end{proof}

\begin{lemma}\label{the: vertex force image beating}
  Given any oriented graph \(G\) and condensation \(f\),
  for any two distinct components \(A\) and \(B\),
  \begin{itemize}
    \item
      for all \(a \in A, b \in B\), if \(a\) ties \(b\),
      then \(f(a)\) ties \(f(b)\).
    \item
      for all \(a \in A, b \in B\), if \(a \to b\),
      then \(f(a) \to f(b)\)
  \end{itemize}
\end{lemma}

\begin{proof}
  Because \(A\) and \(B\) are distinct components,
  then by \cref{the: condensation component partition},
  \(A\) and \(B\) are disjoint,
  therefore for any vertex \(a \in A\),
  and any vertex \(b \in B\), \(f(a) \neq f(b)\).

  Given vertex \(a \in A\) ties vertex \(b \in B\).
  Then \(f(a)\) cannot beat \(f(b)\),
  otherwise \(a\) should beat \(b\), by definition of condensation.
  Also, \(f(b)\) cannot beat \(f(a)\),
  otherwise \(b\) should beat \(a\),
  therefore \(f(a)\) ties \(f(b)\).

  Given vertex \(a \in A\) beats vertex \(b \in B\).
  Then \(f(a)\) cannot tie \(f(b)\),
  otherwise \(a\) should tie \(b\)
  Also \(f(b)\) cannot beat \(f(a)\),
  otherwise \(b\) should beat \(a\).
  Therefore \(f(a) \to f(b)\).
\end{proof}

\begin{theorem}\label{the: vertex force beating relation in condensation}
  Given any oriented graph \(G\) and condensation \(f\),
  for any two distinct components \(A\) and \(B\),
  \begin{itemize}
    \item
      if any vertex \(a \in A\) ties any vertex \(b \in B\),
      then \(A\) ties \(B\).
    \item
      if any vertex \(a \in A\) beats any vertex \(b \in B\),
      then \(A \to B\).
  \end{itemize}
\end{theorem}

\begin{proof}
  Because \(A\) and \(B\) are distinct components,
  then by \cref{the: condensation component partition},
  \(A\) and \(B\) are disjoint,
  therefore for any vertex \(a \in A\),
  and any vertex \(b \in B\), \(f(a) \neq f(b)\).

  Given vertex \(a \in A\) ties vertex \(b \in B\),
  by \cref{the: vertex force image beating},
  \(f(a)\) ties \(f(b)\).
  Thus, for every vertex \(a' \in A\),
  and every vertex \(b' \in B\),
  because \(f(a') = f(a)\) and \(f(b') = f(b)\),
  and \(f(a)\) ties \(f(b)\), therefore \(f(a')\) ties \(f(b')\).
  By the definition of condensation, \(a'\) ties \(b'\).
  Therefore, \(A\) ties \(B\).

  Given vertex \(a \in A\) beats vertex \(b \in B\).
  by \cref{the: vertex force image beating},
  \(f(a) \to f(b)\).
  Thus for every vertex \(a' \in A\),
  and every vertex \(b' \in B\),
  because \(f(a') = f(a)\) and \(f(b') = f(b)\),
  and \(f(a) \to f(b)\), therefore \(f(a') \to f(b')\).
  By the definition of condensation, \(a' \to b'\).
  Therefore, \(A\) ties \(B\).
\end{proof}

\begin{corollary}\label{the: components are vertex in condensation}
  Given any oriented graph \(G\) and condensation \(f\),
  for any two distinct components \(A\) and \(B\),
  either \(A \to B\), or \(B \to A\), or \(A\) ties \(B\).
\end{corollary}

\begin{proof}
  Because components are not empty
  (component of \(v\) always contains \(v\) itself),
  we can take \(a \in A\), and \(b \in B\).
  Because \(G\) is an oriented graph,
  then either \(a \to b\), \(b \to a\), or \(a\) ties \(b\).

  According to \cref{the: vertex force beating relation in condensation},
  if \(a \to b\), then \(A \to B\);
  if \(b \to a\), then \(B \to A\);
  if \(a\) ties \(b\), then \(A\) ties \(B\).
\end{proof}

In \cref{fig: behaves like a vertex example},
we show an example of a set of vertices \(\set{b, c, d}\)
that behaves like the vertex \(v\).
We then try to visualize \cref{the: components are vertex in condensation},
with the help of \cref{fig: condensation example},
we find out that graph condensation just takes the idea
in \cref{fig: behaves like a vertex example}
to the next level: all the components behave like vertices.
A condensation looks for those sets of vertices
that behave like a single vertex (components),
and then condenses them into a single vertex.

\begin{figure}
  \centering
  \begin{subfigure}[b]{0.4\linewidth}
    \centering
    \tikz\graph[layered layout, math nodes, level sep=1cm] {
      a -> {b, c, d};
      {b, c, d} -> f;
      {b, c, d} -> e;
      / [draw] // {b, c, d};
    };
    \caption{every vertex either beats \(\set{b, c, d}\)  %chktex 44
      or is beaten by them.}
    \label{fig: behaves like a vertex example: uncondensed} % chktex 24
  \end{subfigure}
  \begin{subfigure}[b]{0.4\linewidth}
    \centering
    \tikz\graph[layered layout, math nodes, level sep=1cm] {
      a -> v;
      v -> f;
      v -> e;
    };
    \caption{vertices \(\set{b, c, d}\) in %chktex 44
      \cref{fig: behaves like a vertex example: uncondensed}
     behaves exactly like \(v\) in this graph.}
  \end{subfigure}
  \caption{\(\set{b, c, d}\) behaves like a vertex.}  %chktex 44
  \label{fig: behaves like a vertex example}  % chktex 24
\end{figure}

Notice that all oriented graph have the
identity condensation and trivial condensation,
but some oriented graphs may not have other condensations
defined on them.

\begin{figure}
  \centering
  \tikz\graph[simple necklace layout, math nodes, node sep=1.75cm] {
      a -> b -> c -> a;
  };
  \caption{an oriented graph with only
    identity condensation and trivial condensation.}
  \label{fig: no condensation example} % chktex 24
\end{figure}

For example, the graph in \cref{fig: no condensation example}
cannot have any condensation defined on it
except for identity condensation and trivial condensation,
because we cannot find any component that is not the whole
graph or a single vertex.
For example, let's look at set \(\set{a, c}\).
Vertex \(b\) beats a vertex in the set (\(b\) beats \(a\)),
and is beaten by a vertex in the set (\(c\) is beaten by \(b\)).
Therefore the set \(\set{a, c}\) is not a component.

Condensation is a relatively strong transformation,
because it preserves many properties of the graph.
We will prove some theorems that will be useful
in later sections, when looking for kings

\begin{lemma}\label{the: shortest path different components lemma}
  Let \(f: G \to H\) be a condensation of oriented graph \(G\)
  and let \(a_0, a_n\) be two vertices in \(G\) from different
  components. If \(P: a_0 \to a_1 \to \cdots \to a_n\) is a shortest
  path from \(a_0\) to \(a_n\), then vertices
  \(a_0, a_1, \ldots, a_n\) are all in different components.
\end{lemma}

\begin{figure}
\centering
  \begin{subfigure}[b]{0.45\linewidth}
  \centering
    \tikz\graph[simple necklace layout, math nodes, node sep=1cm] {
      dots1[as=\ldots] -> a_{p-1} -> a_p
      -> dots2[as=\ldots] -> a_q -> dots3[as=\ldots];
      / [label=left:\(C\), draw] // {a_p, a_q};
    };
    \caption{\(a_p\) and \(a_q\) are in the same component.}
  \end{subfigure}
  \begin{subfigure}[b]{0.45\linewidth}
    \centering
      \tikz\graph[simple necklace layout, math nodes, node sep=1cm] {
        dots1[as=\ldots] -> a_{p-1} -> a_p
        -> dots2[as=\ldots] -> a_q -> dots3[as=\ldots];
        a_{p-1} -> a_q;
        / [label=left:\(C\), draw] // {a_p, a_q};
      };
      \caption{by \cref{the: vertex force beating relation in condensation},
        \(a_{p-1} \to a_q\).}
    \end{subfigure}

  \caption{the path took a ``detour'' at \(a_{p-1}\).}  %chktex 44
  \label{fig: same component the not the shortest path}  % chktex 24
\end{figure}

\begin{proof}
  Suppose not. Take the the smallest \(p\) such that
  there exists \(a_p\) and \(a_q (p < q \leq n)\)
  are in the same component \(C\).

  Since \(a_p\) is the first vertex in the \(P\)
  that belongs to the same component as another vertex in \(P\).
  \(a_{p-1}\) cannot be in the same component as \(a_p\),
  then \(a_{p-1} \notin C\).
  Because \(a_{p-1} \to a_p\)
  and \cref{the: vertex force beating relation in condensation},
  \(a_{p-1} \to C\).
  Thus \(a_{p-1} \to a_q\), and path
  \(a_0 \to a_1 \to \cdots
  \to a_{p-2} \to a_{p-1} \to a_q \to a_{q+1} \to \cdots
  \to a_{n-1} \to a_n\) exists and shorter than \(P\).
  See \cref{fig: same component the not the shortest path}.
  Contradiction.
\end{proof}

\begin{theorem}\label{the: condensation preserves shortest path}
  Given a condensation \(f: G \to H\),
  for any \(a_0, a_n \in V(G)\),
  such that \(a_0\) and \(a_n\) are in two distinct components.
  A shortest path from \(a_0\) to \(a_n\) is
  \(P_G: a_0 \to a_1 \to \cdots \to a_{n-1} \to a_n\) in \(G\)
  if and only if a shortest path from \(P_H: f(a_0)\) to \(f(a_n)\) is
  \(f(a_0) \to f(a_1) \to \cdots \to f(a_{n-1}) \to f(a_n)\)
  in \(H\).
\end{theorem}

\begin{proof}
  Prove \((\Rightarrow )\):
  Suppose \(P_G\) is a shortest path but \(P_H\) is not a
  shortest path in \(H\).
  Then there exists a shorter path \(P_H'\) in \(H\)
  from \(f(a_0)\) to \(f(a_n)\): \(f(a_0)
    \to f(b_1) \to f(b_2) \to \cdots \to f(b_{m-1}) \to f(b_m)
    \to f(a_n)\) \(m < n-1\).
  Because \(a_0\) and \(a_n\) are in distinct components,
  the path \(P_H'\) does not have length 0.
  The path \(a_0 \to b_1 \to b_2 \to \cdots \to b_{m-1} \to b_m \to a_n\)
  exists (by definition of condensation) and is shorter than
  \(P_G\), a contradiction.

  Prove \((\Leftarrow )\):
  Suppose \(P_H\) is a shortest path in \(H\) but \(P_G\) is
  not a shortest path in \(G\)
  Then there exists a shorter path \(P_G'\) in \(G\)
  from \(a_0\) to \(a_n\):
  \(a_0 \to b_1 \to b_2 \to \cdots \to b_{m-1} \to b_m \to a_n\), \(m < n-1\).
  Because of \cref{the: shortest path different components lemma},
  \(a_0, b_1, b_2, \ldots, b_m, a_n\)
  are all in different components.
  Then by \cref{the: vertex force image beating},
  \(f(a_0) \to f(b_1) \to f(b_2) \to \cdots \to f(b_m) \to f(a_n)\)
  exists and is shorter than \(P_H\), contradiction.
\end{proof}

\begin{corollary}\label{the: condensation preserves beating in distinct components}
  Let \(f: G \to H\) be a condensation,
  for all \(a_0, a_n \in G\),
  such that \(a_0\) and \(a_n\) are in two distinct components,
  \(a_0\) beats \(a_n\) by \(n\) steps in \(G\) if and only if
  \(f(a_0)\) beats \(f(a_n)\) by \(n\) steps in \(H\).
\end{corollary}

\begin{proof}
  Result of \cref{the: condensation preserves shortest path}
  and definition of ``beats by n steps''.
\end{proof}

\begin{corollary}\label{the: condensation preserves king}
  Given a condensation \(f: G \to H\),
  vertex \(k\) is a king in \(G\) if and only if
  \begin{itemize}
    \item \(k\) is a king in the
      induced subgraph of the component of \(k\).
    \item \(f(k)\) is a king in \(H\).
  \end{itemize}
\end{corollary}

\begin{proof}
  Denote component of \(k\) as \(C\).
  Denote the induced subgraph of \(C\) as \(G_c\).

  \((\Rightarrow )\):
  Suppose \(k\) is a king in \(G\).
  Then \(k\) beats every vertex in \(V(G)\) by 1 or 2 steps.
  In particular, \(k\) beats every vertex outside of \(C\) by
  1 or 2 steps.
  Hence, by \cref{the: condensation preserves beating in distinct components},
  \(f(k)\) beats every vertex in \(H\) by 1 or 2 steps,
  Therefore \(f(k)\) is a king in \(H\).

  We know \(k\) beats every vertex in \(C\) by 1 or 2 steps in \(G\).
  Let \(v \in C\) and \(v \neq k\).
  We need to show \(k\) beats every vertex in \(G\) by 1 or 2 steps in \(G_C\).
  \begin{itemize}
    \item Case 1: \(k\) beats \(v\) by 1 step, then
      \(k\) beats \(v\) by 1 step in \(G_c\).
    \item Case 2: \(k\) beats \(v\) by 2 steps, then
      there exists \(a \in V(G)\) such that \(k \to a \to v\) in \(G\).
      because \(v, k \in C\) and
      \cref{the: vertex force beating relation in condensation},
      \(a \in C\) (if \(a \notin C\) will cause a contradiction).
      Therefore, \(k\) beats \(v\) by 2 steps in \(G_c\).
  \end{itemize}
  Thus, \(k\) is a king in \(G_c\),
  and \(f(k)\) is a king in \(H\).

  Prove \((\Leftarrow )\):
  Suppose \(f(k)\) is a king in \(H\),
  and \(k\) is a king in \(G_C\).
  Since \(f(k)\) is a king in \(H\) by 1 or 2 steps,
  \(f(k)\) beats every other vertex \(f(v)\) in \(H\).
  Thus, \(k\) beats every other vertex that is not in \(C\)
  by 1 or 2 steps, by \cref{the: condensation preserves beating in distinct components}.
  Since \(k\) is a king in \(G_c\),
  \(k\) beats every vertex in \(C\) by 1 or 2 steps.
  Therefore, \(k\) beats every vertex in \(V(G)\) by 1 or 2 steps,
  so \(k\) is a king in \(G\).
\end{proof}

Although condensation is a very strong transformation,
some information in the graph does get lost via this transformation.
For example, the beating relationships between all the vertices
in the same components are lost.


\begin{figure}
\centering
  \begin{subfigure}[b]{0.45\linewidth}
  \centering
    \tikz\graph[simple necklace layout, math nodes, node sep=1cm] {
      a_1; a_2;
      a_1 -> {b_1, b_2};
      a_2 -> {b_1, b_2};
      {b_1, b_2} -> c_1;
      {b_1, b_2} -> d_1;
      {a_1, a_2} -> c_1;
      c_1 -> d_1;
      b_1 -> b_2;
      / [label=left:\(A\), draw] // {a_1, a_2};
      / [label=left:\(B\), draw] // {b_1, b_2};
      / [label=right:\(C\), draw] // {c_1};
      / [label=right:\(D\), draw] // {d_1};
    };
    \label{fig: condensation lost information: uncondensed}  % chktex 24
    \caption{the uncondensed oriented graph \(G\).}
  \end{subfigure}
  \begin{subfigure}[b]{0.45\linewidth}
  \centering
    \tikz\graph[simple necklace layout, math nodes, node sep=1cm] {
      A' -> B' -> {C', D'};
      A' -> C' -> D';
    };
    \caption{the condensed graph \(H\).}
    \label{fig: condensation lost information: condensed}  % chktex 24
  \end{subfigure}
  \caption{information lost during condensation.}
  \label{fig: condensation lost information}  % chktex 24
\end{figure}

In \cref{fig: condensation lost information},
we show an example of a condensation,
where the vertices \(a_1, a_2\) are condensed into \(A'\);
\(b_1, b_2\) are condensed into \(B'\);
\(c_1\) is condensed into \(C'\);
and \(d_1\) is condensed into \(D\).
If we just look at the condensed graph,
we notice that we cannot recreate the beating relationship
between \(a_1\) and \(a_2\),
therefore the information about the beating relationship
is lost.
This fact is true for all the vertices in the same component.
Another example in this graph is that
we cannot know the beating relationship between \(b_1\) and \(b_2\)
just by the condensed graph.

Notice that the beating relationship between components
is not lost.
For example, if we want to know the beating relationship
between \(b_1\) and \(a_2\),
we first observe that \(A' \to B'\) in the condensed graph,
then by definition of a condensation,
\(a_2\) has to beat \(b_1\) in the uncondensed graph.
Another example is that we can know that \(a_1\) ties \(d_1\)
because \(A'\) ties \(D'\) in the condensed graph.

Therefore, an ``efficient condensation'' should keep as many
vertices as possible.
One of the most ``efficient'' condensation is the
identity condensation, because it does not lose any information
about this graph.
However, identity condensation also do not mutate the graph
at all, therefore it is not very practical.

\begin{definition}
  Given a set of condensations \(F = \set{f_0, f_1, \ldots, f_n}\)
  where \(f_k: G_k \to H_k\)
  and all the \(V(G_k)\) are equal for \(0 \leq k \leq n\),
  an \keyword{efficient condensation} \(f: G \to H\) in \(F\)
  is a condensation such that
  \(H\) has largest vertex set in
  \(\set{H_0, H_1, H_2, \ldots H_n}\)
\end{definition}

Notice in this definition,
all the \(G_k\)'s where \(0 \leq k \leq n\) are not
necessarily distinct.
One example of efficient condensation is
given a graph \(G\) and all the condensation defined on \(G\),
that is  \(\set{f_0, f_1, \ldots, f_n}\) where \(f_k: G \to H_k\).
Then the identity condensation,
is the only efficient condensation in this set,
since it keeps all the vertices.

Later, we will restrict the types of graphs \(H\) can be.
In so doing, the identity condensation will no longer be allowed.
