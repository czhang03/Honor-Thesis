\section{Tie Component Condensations}

In the previous section we talked about the general property
of graph condensations.
In this section, we will focus on a specific type of condensation
mentioned in previous sections and
\cref{fig: tie components condensation}.
Recall that a tie component is a set of vertices

\begin{definition}
  A \keyword{tie component condensation} is a
  function \(f: Q \to T\),
  where \(Q\) is a quasi-transitive oriented graph,
  \(T\) is a directed graph.
  And \(f\) maps \(V(Q)\) to \(V(T)\) surjectively,
  maps \(E(Q)\) to \(E(T)\) surjectively,
  such that:
  \begin{itemize}
    \item if \(a, b \in V(Q)\) are in the same tie component,
      then \(f(a) = f(b)\).
    \item if \(a, b \in V(Q)\) are in 2 distinct tie components
      \(A, B\) respectively,
      then
      \begin{itemize}
        \item \(f(a) \to f(b)\) if \(A \to B\).
        \item \(f(b) \to f(a)\) if \(B \to A\).
        \item otherwise, tie condensation do not exists.
      \end{itemize}
  \end{itemize}
\end{definition}

\begin{corollary}\label{the: tie condensation are condensation}
  All tie component condensations are graph condensation,
  where the components of the uncondensed graphs
  are the tie components of the uncondensed graphs.
\end{corollary}

\begin{corollary}\label{the: tie condensation results in tournament}
  For any quasi transitive orientated graph \(Q\),
  tie component condensation \(f: Q \to T\) will always exists
  and \(T\) will always be a tournament.
\end{corollary}
\begin{proof}
  By \cref{the: tie component beats tie component} and
  the definition of tie component condensation,
  tie component condensation always exists.

  Also by \cref{the: tie component beats tie component} and
  the definition of tie component condensation,
  for any two distinct vertices \(f(a), f(b) \in T\),
  either \(f(a) \to f(b)\) or \(f(b) \to f(a)\).
  Therefore, \(T\) is a tournament.
\end{proof}

\cref{fig: tie components condensation} is a great example of
tie component condensation.
In this condensation, the function \(g\) maps \(a, b\) to \(A'\),
maps \(c, d, e\) to \(B'\), maps \(f\) to \(C'\),
and maps \(j\) to \(D'\).

It is also helpful to note that the tie components \(C\) and \(D\)
are trivial tie components,
since they only contain vertex \(g\) and \(j\) respectively.

\begin{corollary}
  Given a quasi-transitive digraph \(Q\),
  there only exists a unique tie component condensation
  \(f: Q \to T\)
\end{corollary}
\begin{proof}
  Tie component condensation exists by
  \cref{the: tie condensation results in tournament};
  tie component condensation is unique
  by \cref{the: tie components partition unique}.
\end{proof}

\begin{definition}
  For a given quasi-transitive oriented graph \(Q\),
  we call the result of tie component condensation of \(Q\)
  the \keyword{underlying tournament} of \(Q\).
\end{definition}

It is a nice result that every
quasi-transitive oriented graph
can be condensed into a tournament,
which is one of the most understood family of digraphs.
But just to say that a condensation to tournament exists
is not helpful enough,
since for every graph, there exists a trivial condensation,
which also produce a tournament with only one vertex
(a single vertex is, by definition, a tournament).
Therefore, we need to investigate what we call
the ``efficiency'' of the tie component condensation.

\begin{definition}
  For two orientated graphs \(G, H\), such that
  \begin{itemize}
    \item \(V(G) = V(H)\),
    \item \(a\) ties \(b\) in \(H\) if and only if
      \(a\) ties \(b\) in \(G\),
  \end{itemize}
  then we say \(G, H\) has the same \keyword{tie structure}.
\end{definition}

Graphs with the same tie structure means
that if we draw out all the ties from two graph,
the graph formed by the ties are the same.
In other words, these two graphs are differ
by the orientation of their edges, but not their ties.

\begin{figure}
  \centering
  \begin{subfigure}[b]{0.3\linewidth}
    \centering
    \tikz\graph[simple necklace layout, math nodes, node sep=1cm] {
      a -> b -> c;
      b -> d -> e;
      e -> c -> a;
    };
    \caption{graph \(G\).}
    \label{fig: same tie structure example: G}  % chktex 24
  \end{subfigure}
  \begin{subfigure}[b]{0.3\linewidth}
    \centering
    \tikz\graph[simple necklace layout, math nodes, node sep=1cm] {
      a -> b -> c;
      b -> d -> e;
      e <- c <- a;
    };
    \caption{graph \(H\).}
    \label{fig: same tie structure example: H}  % chktex 24
  \end{subfigure}
  \begin{subfigure}[b]{0.3\linewidth}
    \centering
    \tikz\graph[simple necklace layout, math nodes, node sep=1cm] {
      a; b; c; d; e;
      b --[dashed] e --[dashed] a;
      c --[dashed] d;
    };
    \caption{``tie structures'' of \(G, H\).}
    \label{fig: same tie structure example: tie}  % chktex 24
  \end{subfigure}
  \caption{\(G\) and \(H\) has the same tie.}
  \label{fig: same tie structure example}  % chktex 24
\end{figure}

In \cref{fig: same tie structure example: G} and
\cref{fig: same tie structure example: H},
we show two graph \(G\) and \(H\) with the same tie structure,
In \cref{fig: same tie structure example: tie},
we draw out all the ties from the previous two graphs,
and discover that they are the same.
the only differences between these two graphs are that
\(c \to a\), \(e \to c\) in \(G\),
however \(a \to c\), \(c \to e\) in \(H\),
that is, the only differences between \(G\) and \(H\) are
the orientations of these two edges.

\begin{corollary}\label{the: same tie structure same tie path}
  For \(G\) and \(H\) with the same tie structure,
  if there exists a tie path between \(a, b\) in \(G\),
  then there exists a tie path between \(a, b\) in \(H\).
\end{corollary}

\begin{proof}
  since given any two vertices \(p, q \in G\),
  if \(p\) ties \(q\) in \(G\)
  then \(p\) ties \(q\) in \(H\).
  Therefore if there exists a tie path
  \([a, a_0, a_1, \ldots, a_n, b]\) in \(G\),
  then the same tie path exists in \(H\)
\end{proof}

\begin{theorem}\label{the: tie condensation effcient}
  Given a quasi-transitive graph \(Q\)
  and its tie component condensation \(f\),
  consider the set of all the condensations \(f_k: G_k \to T_k\),
  where \(G_k\) has the same tie structure as \(Q\)
  and \(T_k\) is a tournament:
  \(F = \{f_0, f_1, \ldots, f_{n-1}, f_n\} \).
  \(f\) is an efficient condensation in \(F\).
\end{theorem}

\begin{figure}
  \centering
  \tikz\graph[simple necklace layout, math nodes, node sep=1cm] {
    a --[dashed] dot_1[as=\ldots]
    --[dashed] a_p --[dashed] a_{p+1} --[dashed] dot_2[as=\ldots]
    a_m --[dashed] dots_3[as=\ldots] --[dashed] a_{n-1}
    --[dashed] a_n --[dashed] b;
    / [label=left: component of \(a\), draw] //
    {a, dot_1, a_p};
    / [label=left: another component, draw] //
    {a_{p+1}, dot_2};
  };
  \caption{vertices \(a_p\) and \(a_{p+1}\) crosses components.} %chktex 44
  \label{fig: efficient proof: cross component}  % chktex 24
\end{figure}

\begin{proof}
  Because \(G_k\) are has the same tie structure as \(Q\),
  all \(G_k\)'s have the same vertices.

  If there exists another condensation \(f': G' \to T'\)
  such that \(T'\) has more vertices than
  the underlying tournament \(T\) of \(Q\),
  then there exists
  two vertices \(a, b\) in the same tie component in \(Q\),
  and in different components in \(G'\),
  because the number of components in the uncondensed graph
  is the same as the number of vertices in the condensed
  graph.

  Because \(a, b\) are in the same component in \(Q\),
  there exists a tie path between \(a\) and \(b\) in \(Q\).
  By \cref{the: same tie structure same tie path},
  there exists a tie path between \(a\) and \(b\) in \(G'\).
  Because \(a\) and \(b\) are in different components in \(G'\),
  there exists a point on the tie path between \(a\) and \(b\)
  that ``crosses components'' in \(G'\).

  Formally, denote the tie path between \(a, b\) as:
  \([a, a_0, a_1, \ldots, a_n, b]\),
  then there exists \(a_p\) such that
  \(a_p\) and \(a_{p+1}\) are not in the same component.
  Because \(a_p\) ties \(a_{p+1}\)
  and they are in distinct components,
  then by \cref{the: vertex force image beating},
  \(f'(a)\) ties \(f'(a')\).
  Therefore \(T'\) is not a tournament.
  Contradiction.
\end{proof}

\cref{the: tie condensation effcient} states that
tie component condensation
is not only a most efficient condensation to tournaments
on any quasi-transitive orientated graph,
tie component condensation
is a most efficient in all the condensations to tournaments
defined on all orientated graphs with the same tie structure.

To put it in other way,
for all the orientated graphs with the same tie structure,
quasi-transitive orientated graphs
are the ones that can be condensed into tournaments
most efficiently.
