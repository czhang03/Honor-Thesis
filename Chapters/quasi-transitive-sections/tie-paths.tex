\section{Ties and Tie Paths}

Properties of ties in quasi-transitive digraphs are
vastly important in this chapter.

\begin{lemma}\label{the: arrow direction lemma}
  In a quasi-transitive oriented graph \(G\),
  if vertex \(a\) ties vertex \(b\),
  then for each vertex \(v\) that is adjacent to both \(a\) and \(b\),
  either \(v\) beats both \(a\) and \(b\),
  or \(v\) is beaten by both \(a\) and \(b\)
  (\(\set{a, b} \to v\) or \(v \to \set{a, b}\)).
\end{lemma}

\begin{proof}
  Because \(v\) is adjacent to both \(a\) and \(b\),
  we show all the possible beating relationships
  between \(v\) and \(a, b\) in
  \cref{fig: possible beating between v and a b}.
  Because \(G\) is a quasi-transitive oriented graph,
  if \(a \to v \to b\) or \(b \to v \to a\)
  then \(a, b\) have to be adjacent.
  Therefore, only the possible beating relationships are
  \(\set{a, b} \to v\) or \(v \to \set{a, b}\).
\end{proof}

\begin{figure}
  \centering
  \begin{subfigure}[b]{0.2\linewidth}
    \centering
    \tikz\graph[simple necklace layout, math nodes, node sep=1.75cm] {
      v -> {a, b};
      a --[dashed] b;
    };
    \caption{\(v \to \set{a, b}\).}  % chktex 44
  \end{subfigure}
  \begin{subfigure}[b]{0.2\linewidth}
    \centering
    \tikz\graph[simple necklace layout, math nodes, node sep=1.75cm] {
      v <- {a, b};
      a --[dashed] b
    };
    \caption{\(\set{a, b} \to v\).} % chktex 44
  \end{subfigure}
  \begin{subfigure}[b]{0.2\linewidth}
    \centering
    \tikz\graph[simple necklace layout, math nodes, node sep=1.75cm] {
      v <- a;
      v -> b;
      a --[dashed] b
    };
    \caption{\(a \to v \to b\).}
  \end{subfigure}
  \begin{subfigure}[b]{0.2\linewidth}
    \centering
    \tikz\graph[simple necklace layout, math nodes, node sep=1.75cm] {
      v -> a;
      v <- b;
      a --[dashed] b
    };
    \caption{\(b \to v \to a\).}
  \end{subfigure}
  \caption{all the possible beating relations between
    vertex \(v\) and vertices \(a, b\)}
  \label{fig: possible beating between v and a b}  % chktex 24
\end{figure}

\cref{fig: arrow direction lemma: uncondensed} shows a nice
visualization of \cref{the: arrow direction lemma}.
Consider the set of vertices that are adjacent to both
\(a\) and \(b\).
In this figure the set is
\( \set{d_1, d_2, d_3, s_1, s_2, s_3, s_4} \).
This set is partitioned into 2 parts:
those vertices that dominates both \(a\) and \(b\)
\(\set{d_1, d_2, d_3}\),
and those vertices that are beaten by both \(a\) and \(b\)
\(\set{s_1, s_2, s_3, s_4}\)
The behavior of \(\set{a, b}\) is somewhat like a vertex,
see \cref{fig: arrow direction lemma: condensed}.

\begin{figure}
  \centering
  \begin{subfigure}[b]{0.45\linewidth}
    \centering
    \tikz\graph[layered layout, grow=right, math nodes, level sep=2cm] {
      d_1 -> {a, b} -> s_1;
      d_2 -> {a, b} -> s_2;
      d_3 -> {a, b} -> s_3;
      {a, b} -> s_4;
      / [draw] // {a, b};
    };
    \caption{all the vertices are adjacent to \(a, b\).}
    \label{fig: arrow direction lemma: uncondensed}  % chktex 24
  \end{subfigure}
  \begin{subfigure}[b]{0.45\linewidth}
    \centering
    \tikz\graph[layered layout, grow=right, math nodes, level sep=2cm] {
      d_1 -> v -> s_1;
      d_2 -> v -> s_2;
      d_3 -> v -> s_3;
      v -> s_4;
    };
    \caption{the set \(\set{a, b}\) behaves like \(v\).}  %chktex 44
    \label{fig: arrow direction lemma: condensed}  % chktex 24
  \end{subfigure}
  \caption{only look at all the vertices adjacent
  to both \(a\) and \(b\),
  then \(\set{a, b}\) behaves like a vertex.}
  \label{fig: arrow direction lemma}  % chktex 24
\end{figure}

However, this nice visualization in
\cref{fig: arrow direction lemma} has a very strong prerequisite,
that is, all the vertices need to be adjacent to
both \(a\) and \(b\).
What will happen when we add another vertex that
ties either \(a\) or \(b\) and is adjacent to \(v\)?

\begin{figure}
  \centering
  \begin{subfigure}[b]{0.45\linewidth}
    \centering
    \tikz\graph[simple necklace layout, math nodes, node sep=2cm] {
      v <- {a, b};
      a --[dashed] b --[dashed] c;
      v --c;
    };
    \caption{add a vertex that \(c\) ties \(b\), but not \(v\).}
    \label{fig: add tie to arrow direction lemma: start} % chktex 24
  \end{subfigure}
  \begin{subfigure}[b]{0.45\linewidth}
    \centering
    \tikz\graph[simple necklace layout, math nodes, node sep=2cm] {
      v <- {a, b};
      a --[dashed] b --[dashed] c;
      v <- c;
    };
    \caption{by \cref{the: arrow direction lemma}, \(c \to v\).}
    \label{fig: add tie to arrow direction lemma: finish} % chktex 24
  \end{subfigure}
  \caption{ties are ``transmitting'' arrow directions.}
  \label{fig: add tie to arrow direction lemma}  % chktex 24
\end{figure}

In \cref{fig: add tie to arrow direction lemma: start}
we add a vertex \(c\) that ties \(b\),
but needs to be adjacent to \(v\)
(we use an edge without arrow to denote adjacency).
Because \(b\), \(c\) are both adjacent to \(v\),
and \(c\) ties \(b\),
therefore either \(\set{b, c} \to v\) or \(v \to \set{b, c}\).
However, in the graph, \(b \to v\), therefore \(c \to v\).
Thus, we get \cref{fig: add tie to arrow direction lemma: finish}.

We find out that because of the ties between \(a, b, c\),
the arrow direction between \(v, a\) and \(v, b\),
got ``transmitted'' to \(v, c\)
via the \cref{the: arrow direction lemma}.
We formalize this ``arrow transmission'' idea:

\begin{definition}
  In a digraph, a \keyword{tie path} from
  vertex \(a_0\) to vertex \(a_n\),
  or a tie path between vertices \(a_0\) and \(a_n\),
  is a sequence of vertices
  \([a_0, a_1, a_2, \ldots, a_{n-1}, a_n]\),
  such that for all \(0 \leq k < n\), \(a_k\) ties \(a_{k + 1}\).
  If there is \(n + 1\) vertices in that sequence,
  we will say \keyword{the length of the tie path} is \(n\).
  Note that all the \(a_i\)'s do not have to be distinct.
\end{definition}

A tie path is similar to a path
(called a ``walk'' in~\cite{west_introduction_2001})
in an undirected graph;
just substitute all the edges in a path with ties.
See \cref{fig: tie path and path}.


\begin{figure}
  \centering
  \begin{subfigure}[b]{0.45\linewidth}
    \centering
    \tikz\graph[layered layout, math nodes, level sep=1cm] {
      a --b --c --d
    };
    \caption{a path from \(a\) to \(d\).}
  \end{subfigure}
  \begin{subfigure}[b]{0.45\linewidth}
    \centering
    \tikz\graph[layered layout, math nodes, level sep=1cm] {
      a --[dashed] b --[dashed] c --[dashed] d
    };
    \caption{a tie path from \(a\) to \(d\).}
  \end{subfigure}
  \caption{tie paths are just like paths.}
  \label{fig: tie path and path}  % chktex 24
\end{figure}

Tie path are useful in analyzing quasi-transitive
oriented graphs:

\begin{lemma}\label{the: tie path connection lemma}
  In a digraph,
  if there exists a tie path from vertex \(a\) to vertex \(b\)
  and there exists a tie path from vertex \(b\) to vertex \(c\),
  then there exists a tie path from \(a\) to \(c\).
\end{lemma}
\begin{proof}
  Write the tie path from \(a\) to \(b\) as:
  \([a, a_0, a_1, \ldots, a_n, b]\),
  and the tie path from \(b\) to \(c\) as:
  \([b, b_0, b_1, \ldots, b_m, c]\).
  Then there exists a tie path
  \([a, a_0, a_1, \ldots, a_n, b, b_0, b_1, \ldots, b_m, c]\)
  from \(a\) to \(c\).
  See \cref{fig: tie path connection}.
  \begin{figure}
    \centering
    \tikz\graph[simple necklace layout, math nodes, node sep=0.5cm] {
      a --[dashed] a_0 --[dashed] a_1 --[dashed] dots_a [as=\ldots]
      --[dashed] a_n --[dashed] b --[dashed] b_0 --[dashed]
      b_1 --[dashed] dots_b [as=\ldots] --[dashed] b_m  --[dashed] c;
    };
    \caption{tie path from \(a\) to \(b\), and from \(b\) to \(c\).}
    \label{fig: tie path connection}  % chktex 24
  \end{figure}
\end{proof}

\begin{lemma}\label{the: tie path division lemma}
  In a digraph,
  if there exists a tie path \([a_0, a_1, \ldots, a_{n-1}, a_n]\),
  then for all \(p \neq q\) \(0 \leq p \leq n\) and \(0 \leq q \leq n\),
  there exists a tie path between \(p\) and \(q\).
\end{lemma}
\begin{proof}
  Without loss of generality, assume \(p < q\).
  Then we can find a tie path
  \([a_p, a_{p+1}, \ldots, a_{q-1}, a_q]\),
  see \cref{fig: tie path division}.
  \begin{figure}
    \centering
    \tikz\graph[simple necklace layout, math nodes, node sep=0.5cm] {
      a_0 --[dashed] a_1 --[dashed] dots_a[as=\ldots]
      --[dashed] a_{p-1} --[dashed] a_p --[dashed]
      a_{p+1} --[dashed] dots_b[as=\ldots] --[dashed] a_{q-1}
      --[dashed] a_q --[dashed] a_{q+1} --[dashed] dots_c[as=\ldots]
      --[dashed] a_{n-1} --[dashed] a_{n};
      / [label = left:tie path from \(a_p\) to \(a_q\), draw] //
        {a_p, a_{p+1}, dots_b, a_{q-1}, a_q};
    };
    \caption{the tie path from \(a_0\) to \(a_n\).}
    \label{fig: tie path division}  % chktex 24
  \end{figure}
\end{proof}

After looking at
\cref{fig: add tie to arrow direction lemma},
we can hypothesize that tie paths can transmit
arrow directions across the whole path
in quasi-transitive oriented graph.
See \cref{fig: tie path transmit arrow}.

\begin{figure}
  \centering
  \begin{subfigure}[b]{0.45\linewidth}
    \centering
    \tikz\graph[simple necklace layout, math nodes, node sep=1cm] {
      a --[dashed] b --[dashed] "\ldots" --[dashed] c --[dashed] d;  % chktex 18
      ""; % chktex 18
      a <- f;
      {b, "\ldots", c, d} --f;  % chktex 18
      ""; ""; % chktex 18
    };
    \caption{\(a\) is beaten by \(f\),
      and other vertices adjacent to \(f\)}
  \end{subfigure}
  \begin{subfigure}[b]{0.45\linewidth}
    \centering
    \tikz\graph[simple necklace layout, math nodes, node sep=1cm] {
      a --[dashed] b --[dashed] "\ldots" --[dashed] c --[dashed] d;  % chktex 18
      ""; % chktex 18
      {a, b, "\ldots", c, d} <- f;  % chktex 18
      ""; ""; % chktex 18
    };
    \caption{the arrow direction is
      ``transmitted'' through the tie path}
  \end{subfigure}
  \caption{tie path transmits the arrow direction.}
  \label{fig: tie path transmit arrow}  % chktex 24
\end{figure}

\begin{lemma}\label{the: tie transimission}
  For every tie path in a quasi-transitive oriented graph,
  if vertex \(v\) is adjacent to all the vertices
  on the tie path,
  then \(v\) beats every vertex on the tie path,
  or \(v\) is beaten by every vertex on the tie path.
\end{lemma}

\begin{figure}
  \centering
  \begin{subfigure}[b]{0.45\linewidth}
    \centering
    \tikz\graph[simple necklace layout, math nodes, node sep=1cm] {
      a_1 --[dashed] a_2 --[dashed] "\ldots" % chktex 18
       --[dashed] a_n --[dashed] a_{n+1};
      ""; % chktex 18
      {a_1, a_2, "\ldots", a_n} <- v;  % chktex 18
      a_{n+1} --v;
      ""; ""; % chktex 18
    };
    \caption{\(a_{n+1}\) is adjacent to \(v\).}  % chktex 44
  \end{subfigure}
  \begin{subfigure}[b]{0.45\linewidth}
    \centering
    \tikz\graph[simple necklace layout, math nodes, node sep=1cm] {
      a_1 --[dashed] a_2 --[dashed] "\ldots" % chktex 18
       --[dashed] a_n --[dashed] a_{n+1};
      ""; % chktex 18
      {a_1, a_2, "\ldots", a_n, a_{n+1}} <- v;  % chktex 18
      ""; ""; % chktex 18
    };
    \caption{then \(a_{n+1}\) has to be beaten by \(v\).}  % chktex 44
  \end{subfigure}
  \caption{the induction step of the proof of
    \cref{the: tie transimission}}
  \label{fig: tie transimission proof}  % chktex 24
\end{figure}

\begin{proof}
  We can prove this lemma via induction,
  follow the intuition in
  \cref{fig: add tie to arrow direction lemma}.

  Start with a tie path of length 1,
  then we have \(a_0\) ties \(a_1\) in this tie path,
  and \(v\) is adjacent to both of them.
  \cref{the: arrow direction lemma} proves the result
  for path of length 1.

  Assume the property holds for
  any tie path of length \(n\) and vertex \(v\),
  then we need to prove this property holds for
  any tie path of length \(n + 1\).
  See \cref{fig: tie transimission proof},
  take a tie path \([a_0, a_1, \ldots, a_n, a_{n+1}]\).
  we can apply the induction hypothesis to the
  tie path \([a_0, a_1, \ldots, a_n]\).
  Because \(a_n\) ties \(a_{n+1}\), and
  \(v\) is adjacent to both of them.
  Therefore,
  if \(v \to {a_0, \ldots, a_n}\), then \(v \to a_{n+1}\),
  then \(v\) beats the whole tie path;
  otherwise \({a_0, \ldots, a_n} \to v\), then \(a_{n+1} \to v\),
  then \(v\) is beaten by the whole tie path.
  Therefore this lemma still holds for tie path of length \(n + 1\).
\end{proof}

\cref{the: tie transimission} generalizes
\cref{the: arrow direction lemma},
however, it is still far from elegant,
because we still requires the vertex \(v\)
to be adjacent to the whole tie path.
