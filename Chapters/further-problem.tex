\chapter{Further Problems}

In \cref{chap: semi-complete digraph} and \cref{chap: general oriented graph},
we see that because of the addition of two types of ties,
we can construct many more \((n, k)\) semi-complete digraphs
or \((n, k)\) oriented graphs than \((n, k)\) tournaments.
Then what can we construct if we limit the number of ties
or double ties?

\begin{definition}
  a \((n, k, t)\) digraph is a digraph with \(n\) vertices,
  \(k\) kings and \(t\) ties (or double ties).
\end{definition}

In \cref{chap: semi-complete digraph}, and \cref{chap: general oriented graph},
we explore the construction of \((n, k)\) semi-complete digraphs
and \((n, k)\) oriented graphs with 1 tie.
In other word, we have solved the problem of constructing
\((n, k, 1)\) oriented graph and semi-complete digraphs.
What happens with \(t\) gets larger?

One of the useful thing to note when approaching the
problem of the existence of \((n, k, t)\) digraphs
(or oriented graphs, semi-complete digraphs, etc.)
in the end of \cref{sec: quasi-transitive king},
we give a way to construct a quasi-transitive oriented graph
with given number of kings.
This way of constructing quasi-transitive oriented graphs
may be very useful,
since we can manipulate the number of ties in the graph
without changing the number of vertices and kings.
For a tie components with \(n\) vertices,
the number of ties in the component can be any number
between \(n - 1\)
(ties forms a spanning tree of the component)
and \(\frac{n(n+1)}{2}\)
(there exists a tie between every two vertices).

In \cref{sec:kings}, we showed that there exists king with very
low out-degrees.
What is the property of a king with very low out-degree?
What is the property of a king with only out-degree 1?
what is the property of a king that has the lowest out-degree
in the graph?
What is the property of a king that has the lowest out-degrees
among all of the kings in the graph?

In \cref{sec: quasi-transitive king},
we have showed that king has very rich structures.
However, \cref{the: king partitions in quasi-transitive}
and \cref{the: D S of king adjacent in quasi-transitive}
are not used in this thesis.
Can we use these 2 property to deduce any property
of quasi-transitive oriented graph with kings?
A interesting path is that
\cref{the: D S of king adjacent in quasi-transitive}
implies in a quasi-transitive oriented graph with king \(k\),
all the tie components are either the subset of \(D_k\) or \(S_k\).

\begin{definition}
  A infinite graph is a graph with infinite number of vertices.
\end{definition}

Can we extend these result to infinite graph?
Can we keep the definition of tie components when
we expand the context to infinite graph?
How will the property of infinite path
and infinite tie path change the results in this thesis?

In~\cite{bang-jensen_kings_1998}
and~\cite{bangjensen_quasitransitive_1995},
the author uses the idea of strong digraph and strong components
(see the definition of strong and strong components
in~\cite{west_introduction_2001})
to deduce the property of quasi-transitive oriented graph.
We can show that given any \emph{non-trivial} condensation,
the uncondensed graph is strong
if and only if its underlying tournament is strong
(\emph{spoiler alert!}:
\cref{the: shortest path different components lemma} will be useful).
Can we combine our results with the result
in~\cite{bang-jensen_kings_1998} and~\cite{bangjensen_quasitransitive_1995}?

In \cref{sec: graph condensation},
we showed that graph condensation is a very useful transformation.
and many graph only have identity condensation
and trivial condensation defined on it.
How hard is a condensation?
How many graph only have identity condensation
and trivial condensation on them?

How useful is graph condensations?
we have showed that they preserves the shortest path
(\cref{the: condensation preserves shortest path}),
and we can show that they preserves the strong property of the graph.
What are other property that graph condensations preserves?
