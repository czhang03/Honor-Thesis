% Copyright 2004 by Till Tantau <tantau@users.sourceforge.net>.
%
% In principle, this file can be redistributed and/or modified under
% the terms of the GNU Public License, version 2.
%
% However, this file is supposed to be a template to be modified
% for your own needs. For this reason, if you use this file as a
% template and not specifically distribute it as part of a another
% package/program, I grant the extra permission to freely copy and
% modify this file as you see fit and even to delete this copyright
% notice.

\documentclass{beamer}

  % packages
  \usepackage{bookmark}
  \usepackage{tikz}
  \usetikzlibrary[graphs, graphdrawing, graphs.standard, intersections]
  \usegdlibrary{circular, layered, trees}
  \usepackage[justification=centering, position=bottom]{subcaption}
  % set the bibliography
  \usepackage[style=authoryear]{biblatex}
  \addbibresource{bibfile.bib}

  % use section title as frame title
  \addtobeamertemplate{frametitle}{
  \let\insertframetitle\insertsectionhead}{}
  \addtobeamertemplate{frametitle}{
  \let\insertframesubtitle\insertsubsectionhead}{}
  \makeatletter
  \CheckCommand*\beamer@checkframetitle{\@ifnextchar\bgroup\beamer@inlineframetitle{}}
  \renewcommand*\beamer@checkframetitle{\global\let\beamer@frametitle\relax\@ifnextchar\bgroup\beamer@inlineframetitle{}}  %chktex 21
  \makeatother


  % There are many different themes available for Beamer. A comprehensive
  % list with examples is given here:
  % http://deic.uab.es/~iblanes/beamer_gallery/index_by_theme.html
  % You can uncomment the themes below if you would like to use a different
  % one:
  %\usetheme{AnnArbor}
  %\usetheme{Antibes}
  %\usetheme{Bergen}
  %\usetheme{Berkeley}
  %\usetheme{Berlin}
  %\usetheme{Boadilla}
  %\usetheme{boxes}
  %\usetheme{CambridgeUS}
  %\usetheme{Copenhagen}
  %\usetheme{Darmstadt}
  \usetheme{default}
  %\usetheme{Frankfurt}
  %\usetheme{Goettingen}
  %\usetheme{Hannover}
  %\usetheme{Ilmenau}
  %\usetheme{JuanLesPins}
  %\usetheme{Luebeck}
  %\usetheme{Madrid}
  %\usetheme{Malmoe}
  %\usetheme{Marburg}
  %\usetheme{Montpellier}
  %\usetheme{PaloAlto}
  %\usetheme{Pittsburgh}
  %\usetheme{Rochester}
  %\usetheme{Singapore}
  %\usetheme{Szeged}
  %\usetheme{Warsaw}

  \title{Kings in Generalized Tournaments}

  % A subtitle is optional and this may be deleted
  % \subtitle{}

  \author{C. Zhang.}
  % - Give the names in the same order as the appear in the paper.
  % - Use the \inst{?} command only if the authors have different
  %   affiliation.

  \institute[Wheaton] % (optional, but mostly needed)
  {
  Department of Mathematics\\
  Wheaton College Student
  }
  % - Use the \inst command only if there are several affiliations.
  % - Keep it simple, no one is interested in your street address.

  \date{Thesis Defense}
  % - Either use conference name or its abbreviation.
  % - Not really informative to the audience, more for people (including
  %   yourself) who are reading the slides online

  \subject{Kings in Generalized Tournaments}
  % This is only inserted into the PDF information catalog. Can be left
  % out.

  % If you have a file called "university-logo-filename.xxx", where xxx
  % is a graphic format that can be processed by latex or pdflatex,
  % resp., then you can add a logo as follows:

  % \pgfdeclareimage[height=0.5cm]{university-logo}{university-logo-filename}
  % \logo{\pgfuseimage{university-logo}}


  % ============================================
  % DOCUMENT START
  % ============================================
\begin{document}

\begin{frame}
  \titlepage{}
\end{frame}

\section{Background}

\subsection{tournaments}

\begin{frame}
  \begin{definition}[\cite{maurer_king_1980}]
    An \textbf{tournament} consists of a set of \textbf{vertices} \(V\),
    and a set of \textbf{edges} \(E\) which consists of ordered pairs of vertices,
    such that
    \begin{itemize}
      \item for all vertex \(a\), \((a, a) \notin E\).
      \item for all vertices \(a, b\), \((a, b) \in E\) or \((b, a) \in E\),
      but not both.
    \end{itemize}
  \end{definition}
  \begin{figure}
    \centering
    \tikz\graph[simple necklace layout, math nodes, node sep=1.75cm] {
    e -> {a, b};
    b -> c -> d -> e;
    a -> d -> b;
    c -> a -> b;
    e -> c
    };
    \caption{example of a tournament.}
  \end{figure}
\end{frame}

\begin{frame}
  If \((a, b)\) in edge set \(E\), we write it as \(a \to b\) or \(a\) beats \(b\).

  \begin{figure}
    \centering
    \tikz\graph[simple necklace layout, math nodes, node sep=1.75cm] {
    e -> {a, b};
    b -> c -> d -> e;
    a -> d -> b;
    c -> a -> b;
    e -> c
    };
    \caption{example of a tournament.}
  \end{figure}
\end{frame}

\subsection{kings}

\begin{frame}
  \begin{definition}[\cite{maurer_king_1980}]
    A \textbf{king} (or \textbf{2-king}) in a oriented graph is a vertex that can beat every vertex by at most 2 steps.
  \end{definition}
  \begin{figure}
    \centering
    \tikz\graph[layered layout, grow=right, sibling distance=2cm,level sep=0.75cm] {
    k -> {a, b};
    b -> c;
    };
  \end{figure}
\end{frame}

\begin{frame}
  The kings in this graph are \(a, d, e\).

  \begin{figure}
    \centering
    \tikz\graph[simple necklace layout, math nodes, node sep=1.75cm] {
    e -> {a, b, c};
    b -> c -> d -> e;
    a -> {b, c, d};
    b -> d;
    };
    \caption{example of a tournament.}
  \end{figure}
\end{frame}

\begin{frame}
  The kings in this graph are \(a, d, e\):

  \begin{figure}
    \centering
    \begin{subfigure}{0.32\linewidth}
      \centering
      \tikz\graph[simple necklace layout, math nodes, node sep=0.75cm] {
      e, a, b, c, d;
      a -> d -> e;
      a -> {b, c};
      };
      \caption{\(a\) is a king.}
    \end{subfigure}
    \begin{subfigure}{0.32\linewidth}
      \centering
      \tikz\graph[simple necklace layout, math nodes, node sep=0.75cm] {
      e, a, b, c, d;
      d -> e -> {a, b, c};
      };
      \caption{\(d\) is a king.}
    \end{subfigure}
    \begin{subfigure}{0.32\linewidth}
      \centering
      \tikz\graph[simple necklace layout, math nodes, node sep=0.75cm] {
      e -> {a, b, c};
      c -> d;
      };
      \caption{\(e\) is a king.}
    \end{subfigure}
  \end{figure}
\end{frame}


\section{Generalization of Tournaments}

\subsection{relation between three generalizations}

\begin{frame}
  \begin{figure}
    \centering
    \begin{tikzpicture}

      \begin{scope}[shift={(3cm,-5cm)}, fill opacity=0.5,
        mytext/.style={text opacity=1, font=\large\bfseries}]

        \draw[draw = black,name path=circle 1] (-1.5,0) circle (3);
        \draw[draw = black,name path=circle 2] (1.5,0) circle (3);

        \pgftransparencygroup
        \clip (-1.5,0) circle (3);
        \fill[gray] (1.5,0) circle (3);
        \filldraw[draw, fill=gray, name intersections={of=circle 1 and circle 2}] (intersection-1) .. controls +(-4,1) and +(-4,-1) ..(intersection-2);
        \endpgftransparencygroup

        \node[mytext] at (-3.8,0) (B) {O};
        \node[mytext] at (3.8,0) (C) {S};
        \node[mytext] at (0,0) (D) {T};
        \node[mytext] at (-2,1) (E) {Q};
      \end{scope}
    \end{tikzpicture}
    \caption{the relationship between tournaments,
    semi-complete digraphs, oriented graphs,
    and quasi-transitive oriented graphs.}
    \label{fig: generalized tournaments relationship}  % chktex 24
  \end{figure}
\end{frame}

\section{Quasi-transitive Oriented Graph}

\subsection{oriented graph}

\begin{frame}
  \begin{definition}
    An \textbf{oriented graph} consists of a set of \textbf{vertices} \(V\),
    and a set of \textbf{edges} \(E\) which consists of ordered pairs of vertices,
    such that
    \begin{itemize}
      \item for all vertex \(a\), \((a, a) \notin E\).
      \item for all vertices \( \{a, b\} \), if \(a\) beats \(b\),
      then \(b\) does not beat \(a\).
    \end{itemize}
  \end{definition}

  \begin{figure}
    \centering
    \tikz\graph[layered layout, grow=right, sibling distance=2cm,level sep=0.75cm] {
    {d, a}
    -> b
    -> c
    -> a;
    };
    \caption{example of an oriented graph.}
  \end{figure}
\end{frame}

\begin{frame}
  \begin{definition}
    A vertex \(a\) is \textbf{adjacent} to vertex \(b\) if \(a \to b\) or \(b \to a\).
  \end{definition}
  \begin{definition}
    There exists a \textbf{tie} between \(a\) and \(c\), if \(a\) is not adjacent to \(c\).
  \end{definition}

  \begin{figure}
    \centering
    \tikz\graph[layered layout, math nodes, grow=right, sibling distance=2cm,level sep=0.75cm] {
    {a, c}
    -> [bend right] d
    -> b
    -> [bend right] {c, a};
    };
    \caption{another example of an oriented graph.}
  \end{figure}
\end{frame}


\begin{frame}
  King in this graph is \(d, e\):

  \begin{figure}
    \centering
    \tikz\graph[simple necklace layout, math nodes, node sep=1.75cm] {
    e -> {a, b, c};
    b -> c -> d -> e;
    a -> d;
    b -> a;
    };
    \caption{an example of kings.}
  \end{figure}
\end{frame}


\subsection{definition}

\begin{frame}
  \begin{definition}[\cite{bangjensen_quasitransitive_1995}]
    \textbf{Quasi-transitive} oriented graph is an oriented graph such that
    if \(a \to b\) and \(b \to c\) then \(a\) and \(c\) are adjacent.
  \end{definition}
  \begin{figure}
    \centering
    \tikz\graph[layered layout, math nodes] {
    a_1 -> b_1 -> c_1 -> a_1;
    a_2 -> b_2 -> c_2;
    a_2 -> c_2;
    };
    \caption{examples of quasi-transitive oriented graphs.}
  \end{figure}
\end{frame}

\subsection{ties}

\begin{frame}
  \begin{lemma}
    In Quasi-transitive oriented graphs, If vertex \(a\) ties with vertex \(b\),
    then for any other vertex \(v\) that does not tie with \(a\) or \(b\),
    \(v \to \{a, b\} \) or \( \{a, b\} \to v\)
  \end{lemma}
  \begin{figure}
    \centering
    \begin{subfigure}{0.45\linewidth}
      \centering
      \tikz\graph[simple necklace layout, math nodes, node sep=1cm] {
      v <- {a, b};
      a --[dashed] b;
      };
      \caption{\(v \to \{a, b\} \).}
    \end{subfigure}
    \begin{subfigure}{0.45\linewidth}
      \centering
      \tikz\graph[simple necklace layout, math nodes, node sep=1cm] {
      v -> {a, b};
      a --[dashed] b;
      };
      \caption{\( \{a, b\} \to v\).}
    \end{subfigure}
    \caption{an example where \(a\) ties with \(b\).}
  \end{figure}
\end{frame}

\subsection{tie transmits arrows}
\begin{frame}
  \begin{figure}
    \centering
    \begin{subfigure}{0.3\linewidth}
      \centering
      \tikz\graph[simple necklace layout, math nodes, node sep=1cm] {
      v --{a, b};
      a --[dashed] b;
      };
      \caption{both \(a\) and \(b\) are adjacent to \(v\).}
    \end{subfigure}
    \begin{subfigure}{0.3\linewidth}
      \centering
      \tikz\graph[simple necklace layout, math nodes, node sep=1cm] {
      v <- a;
      v --b;
      a --[dashed] b;
      };
      \caption{let \(a \to v\).}
    \end{subfigure}
    \begin{subfigure}{0.3\linewidth}
      \centering
      \tikz\graph[simple necklace layout, math nodes, node sep=1cm] {
      v <- {a, b};
      a --[dashed] b;
      };
      \caption{then \(b\) has to beat \(v\).}
    \end{subfigure}
    \caption{arrow direction transmitted from \(a\) to \(b\).}
  \end{figure}
\end{frame}

\begin{frame}
  \begin{figure}
    \centering
    \begin{subfigure}{0.45\linewidth}
      \centering
      \tikz\graph[simple necklace layout, math nodes, node sep=1cm] {
      v <- a;
      };
      \caption{starts with vertex \(a\).}
    \end{subfigure}
    \begin{subfigure}{0.45\linewidth}
      \centering
      \tikz\graph[simple necklace layout, math nodes, node sep=1cm] {
      v <- {a, b};
      a --[dashed] b;
      };
      \caption{add a vertex \(b\) that ties \(a\).}
    \end{subfigure}
    \begin{subfigure}{0.45\linewidth}
      \centering
      \tikz\graph[simple necklace layout, math nodes, node sep=1cm] {
      v <- {a, b, c};
      a --[dashed] b --[dashed] c;
      };
      \caption{add another vertex \(c\) ties \(b\).}
    \end{subfigure}
    \begin{subfigure}{0.45\linewidth}
      \centering
      \tikz\graph[simple necklace layout, math nodes, node sep=1cm] {
      v <- {a, b, c, d};
      a --[dashed] b --[dashed] c --[dashed] d;
      };
      \caption{add another vertex \(d\) ties \(c\).}
    \end{subfigure}
    \caption{tie is transmitting the direction of arrow.}
  \end{figure}
\end{frame}

\subsection{tie paths}

\begin{frame}
  \begin{definition}
    In a digraph, a \textbf{tie path} from
    vertex \(a_0\) to vertex \(a_n\),
    or a tie path between vertices \(a_0\) and \(a_n\),
    is a sequence of vertices
    \([a_0, a_1, a_2, \ldots, a_{n-1}, a_n]\),
    such that for all \(0 \leq k < n\), \(a_k\) ties \(a_{k + 1}\).
    Note that all the \(a_i\)'s do not have to be distinct.
  \end{definition}

  \begin{figure}
    \centering
    \tikz\graph[layered layout, grow=right, math nodes, sibling distance=0.75cm] {
    a --[dashed] b --[dashed] c --[dashed] d
    };
    \caption{a tie path from \(a\) to \(b\).}
  \end{figure}
\end{frame}

\begin{frame}
  \begin{figure}
    \centering
    \begin{subfigure}{0.45\linewidth}
      \centering
      \tikz\graph[simple necklace layout, math nodes, node sep=1cm] {
      a -> b -> c -> d -> e;
      e -> {b, c};
      };
    \end{subfigure}
    \begin{subfigure}{0.45\linewidth}
      \centering
      \tikz\graph[simple necklace layout, math nodes, node sep=1cm] {
      a -> b -> c -> d -> e;
      b --[dashed] d --[dashed] a --[dashed] c;
      a --[dashed] e;
      e -> {b, c};
      };
    \end{subfigure}
    \caption{tie path example.}
  \end{figure}
\end{frame}

\begin{frame}
  \begin{lemma}
    In Quasi-transitive oriented graphs, If vertex \(a\) ties with vertex \(b\),
    then for any other vertex \(v\) that does not tie with \(a\) or \(b\),
    \(v \to \{a, b\} \) or \( \{a, b\} \to v\)
  \end{lemma}

  \begin{lemma}
    For every tie path in a quasi-transitive oriented graph,
    if vertex \(v\) is adjacent to all the vertices
    on the tie path,
    then \(v\) beats every vertex on the tie path,
    or \(v\) is beaten by every vertex on the tie path.
  \end{lemma}
\end{frame}

\begin{frame}
  \begin{lemma}
    For every tie path in a quasi-transitive oriented graph,
    if vertex \(v\) is adjacent to all the vertices
    on the tie path,
    then \(v\) beats every vertex on the tie path,
    or \(v\) is beaten by every vertex on the tie path.
  \end{lemma}

  \begin{figure}
    \centering
    \begin{subfigure}{0.45\linewidth}
      \centering
      \tikz\graph[simple necklace layout, math nodes, node sep=0.75cm] {
      v <- {a, b, c, d};
      a --[dashed] b --[dashed] c --[dashed] d
      };
    \end{subfigure}
    \begin{subfigure}{0.45\linewidth}
      \centering
      \tikz\graph[simple necklace layout, math nodes, node sep=0.75cm] {
      v -> {a, b, c, d};
      a --[dashed] b --[dashed] c --[dashed] d
      };
    \end{subfigure}
    \caption{a tie path from \(a\) to \(d\).}
  \end{figure}
\end{frame}

\subsection{tie components}
\begin{frame}
  \begin{figure}
    \centering
    \begin{subfigure}[b]{0.45\linewidth}
      \centering
      \tikz\graph[simple necklace layout, math nodes, node sep=0.5cm] {
      b <- a;
      / [label = left:\(C\), draw] // {a};
      };
      \caption{\(a\) beats \(b\).}
    \end{subfigure}
    \begin{subfigure}[b]{0.45\linewidth}
      \centering
      \tikz\graph[simple necklace layout, math nodes, node sep=0.5cm] {
      a --[dashed] c;
      {a, c} -> b;
      / [label = left:\(C\), draw] // {a, c};
      };
      \caption{direction transmitted from \(a\) to \(c\).}
    \end{subfigure}
    \begin{subfigure}[b]{0.45\linewidth}
      \centering
      \tikz\graph[simple necklace layout, math nodes, node sep=0.5cm] {
      a --[dashed] c --[dashed] d;
      ""; "";  % chktex 18
      a -> b;
      c -> b;
      d -> b;
      / [label = left:\(C\), draw] // {a, c, d};
      };
      \caption{transmitted from \(a\) to \(c\) to \(d\).}
    \end{subfigure}
    \begin{subfigure}[b]{0.45\linewidth}
      \centering
      \tikz\graph[simple necklace layout, math nodes, node sep=0.5cm] {
      a --[dashed] c --[dashed] d;
      c --[dashed] e;
      "";  % chktex 18
      a -> b;
      c -> b;
      d -> b;
      e -> b;
      / [label = left:\(C\), draw] // {a, c, d, e};
      };
      \caption{transmitted from \(a\) to \(c\) to \(e\).}
    \end{subfigure}
    \caption{the arrow direction to \(b\) was transmitted in tie component \(C\).}
    \label{fig: transmitting arrow direction in tie component}  % chktex 24
  \end{figure}
\end{frame}

\begin{frame}
  \begin{definition}
    In a digraph \(G\), a
    \textbf{tie component of vertex \(a\)}:
    \(C(a)\), all vertices \(v\)
    such that there exists a tie path between \(a\) and \(v\).
  \end{definition}

  \begin{definition}
    In a digraph \(G\), a tie component \(C(a)\)
    is a \textbf{trivial tie component}
    if \(C(a)\) only contains \(a\) itself
    and no other vertex.
  \end{definition}

  \begin{corollary}\label{the: adjacent if not in component}
    In a digraph, if vertex \(v\) is not in tie component \(C(a)\),
    then \(v\) is adjacent to \(C(a)\).
  \end{corollary}
\end{frame}

\begin{frame}

  \begin{lemma}\label{the: component and a single point}
    In a quasi-transitive oriented graph,
    for any tie component \(C(a)\) and
    any vertex \(v\) such that \(v \notin C(a)\),
    then either \(C(a) \to v\) or \(v \to C(a)\).
  \end{lemma}

  \begin{figure}
    \centering
    \begin{subfigure}{0.4\linewidth}
      \centering
      \tikz\graph[layered layout, grow=right, math nodes, level sep = 0.75cm] {
      / [label=\(C(a)\), draw] // [simple necklace layout, node sep = 0.3cm]
      {
      a --[dashed] b --[dashed] {c, d};
      d --[dashed] c;
      } <- v;
      };
      \caption{\(v \to C(a)\).}
    \end{subfigure}
    \begin{subfigure}{0.4\linewidth}
      \centering
      \tikz\graph[layered layout, grow=right, math nodes, level sep = 0.75cm] {
      / [label=\(C(a)\), draw] // [simple necklace layout, node sep = 0.3cm]
      {
      a --[dashed] b --[dashed] {c, d};
      d --[dashed] c;
      } -> v;
      };
      \caption{\(C(a) \to v\).}
    \end{subfigure}
  \end{figure}

\end{frame}

\begin{frame}
  Consider tie component \(C(v)\). Without loss of generosity, assume \(v \to C(a)\)

  \begin{figure}
    \centering
    \begin{subfigure}{0.45\linewidth}
      \centering
      \tikz\graph[simple necklace layout, grow=right, math nodes, node sep = 0.75cm] {
      v -> "C(a)" [draw, circle]; % chktex 36 % chktex 18
      };
      \caption{direction is \(v \to C(a)\).}
    \end{subfigure}
    \begin{subfigure}{0.45\linewidth}
      \centering
      \tikz\graph[simple necklace layout, math nodes, node sep = 0.75cm] {
      v --[dashed] p;
      {v, p} -> "C(a)" [draw, circle]; % chktex 36 % chktex 18
      };
      \caption{transmitted from \(v\) to \(p\).}
    \end{subfigure}
    \begin{subfigure}{0.45\linewidth}
      \centering
      \tikz\graph[simple necklace layout, math nodes, node sep = 0.75cm] {
      v --[dashed] p;
      v --[dashed] q;
      {v, p, q} -> "C(a)" [draw, circle]; % chktex 36 % chktex 18
      / [label=left:\(C(v)\), draw] // {v, p, q};
      };
      \caption{transmitted from \(v\) to \(q\).}
    \end{subfigure}
    \begin{subfigure}{0.45\linewidth}
      \centering
      \tikz\graph[simple necklace layout, grow=right, math nodes, node sep = 0.75cm] {
      "C(v)" [draw, circle] -> "C(a)" [draw, circle]; % chktex 36 % chktex 18
      };
      \caption{\(C(v) \to C(a)\).}
    \end{subfigure}
  \end{figure}
\end{frame}

\subsection{condensation to tournaments}

\begin{frame}
  \begin{theorem}\label{the: tie component beats tie component}
    For any two distinct tie components \(C(v)\) and \(C(a)\)
    in a quasi-transitive oriented graph,
    \(C(v) \to C(a)\) or \(C(a) \to C(v)\).
  \end{theorem}
\end{frame}

\begin{frame}
  \begin{figure}
    \centering
    \begin{subfigure}[b]{0.45\linewidth}
      \centering
      \tikz\graph[simple necklace layout, math nodes, node sep=0.5cm] {
      a; b;
      a -> {c, d, e};
      b -> {c, d, e};
      {c, d, e} -> f;
      {a, b} -> f;
      j -> {c, d, e};
      {a, b} -> j;
      f -> j;
      c -> d --[dashed] e --[dashed] c;
      a --[dashed] b;
      };
      \caption{a quasi-transitive oriented graph.}
    \end{subfigure}
    \begin{subfigure}[b]{0.45\linewidth}
      \centering
      \tikz\graph[simple necklace layout, math nodes, node sep=0.4cm] {
      a; b; c; d; e; f; j;
      d --[dashed] e --[dashed] c;
      a --[dashed] b;
      / [label = left:\(A\), draw, circle] // {a, b};
      / [label = left:\(B\), draw] // {d, e, c};
      / [label = right:\(C\), draw, circle] // {f};
      / [label = right:\(D\), draw, circle] // {j};
      };
      \caption{find its tie components.}
    \end{subfigure}
    \begin{subfigure}[b]{0.45\linewidth}
      \centering
      \tikz\graph[simple necklace layout, math nodes, node sep=0.4cm] {
      A [draw, circle, minimum size=0.75cm] -> B [draw, circle, minimum size=0.9cm]
      -> C [draw, circle, minimum size=0.5cm] -> D [draw, circle, minimum size=0.5cm];
      A -> D -> B;  % chktex 13
      A -> C;  % chktex 13
      };
      \caption{components beat each other.}
    \end{subfigure}
    \begin{subfigure}[b]{0.45\linewidth}
      \centering
      \tikz\graph[simple necklace layout, math nodes, node sep=0.5cm] {
      A' -> B' -> C' -> D';
      A' -> D' -> B';
      A' -> C';
      };
      \caption{components are just like vertices.}
    \end{subfigure}
  \end{figure}
\end{frame}

\begin{frame}
  \begin{theorem}
    Tie component condensation of given
    quasi-transitive oriented graph \(Q\)
    will always result in a unique tournament \(T\),
    where \(T\) is called the
    \textbf{underlying tournament of \(Q\)}.
  \end{theorem}
  \begin{figure}
    \centering
    \begin{subfigure}[b]{0.45\linewidth}
      \centering
      \tikz\graph[simple necklace layout, math nodes, node sep=0.5cm] {
      a_0; a_1;
      a_0 -> {b_0, b_1, b_2};
      a_1 -> {b_0, b_1, b_2};
      {b_0, b_1, b_2} -> c;
      {a_0, a_1} -> d;
      d -> {b_0, b_1, b_2};
      {a_0, a_1} -> d;
      c -> d;
      b_0 -> b_1 --[dashed] b_2 --[dashed] b_0;
      a_0 --[dashed] a_1;
      };
      \caption{quasi-transitive oriented graph \(Q\).}
    \end{subfigure}
    \begin{subfigure}[b]{0.45\linewidth}
      \centering
      \tikz\graph[simple necklace layout, math nodes, node sep=0.5cm] {
      A' -> B' -> C' -> D';
      A' -> D' -> B';
      A' -> C';
      };
      \caption{underlying tournament of \(Q\).}
    \end{subfigure}
  \end{figure}
\end{frame}

\subsection{graph condensation}

\begin{frame}

  \begin{figure}
    \centering
    \begin{subfigure}[b]{0.45\linewidth}
      \centering
      \tikz\graph[simple necklace layout, math nodes, node sep=0.5cm] {
      a; b;
      a --[dashed] {c, d, e};
      b --[dashed] {c, d, e};
      {c, d, e} -> f;
      {a, b} -> f;
      j -> {c, d, e};
      {a, b} -> j;
      f -> j;
      c -> d -> e --[dashed] c;
      a -> b;
      };
      \caption{an oriented graph \(G\).}
    \end{subfigure}
    \begin{subfigure}[b]{0.45\linewidth}
      \centering
      \tikz\graph[simple necklace layout, math nodes, node sep=0.4cm] {
      a; b; c; d; e; f; j;
      d -> e --[dashed] c;
      c -> d;
      a -> b;
      / [label = left:\(A\), draw, circle] // {a, b};
      / [label = left:\(B\), draw] // {d, e, c};
      / [label = right:\(C\), draw, circle] // {f};
      / [label = right:\(D\), draw, circle] // {j};
      };
      \caption{components of \(G\).}
    \end{subfigure}
    \begin{subfigure}[b]{0.45\linewidth}
      \centering
      \tikz\graph[simple necklace layout, math nodes, node sep=0.4cm] {
      A [draw, circle, minimum size=0.75cm] --[dashed]
      B [draw, circle, minimum size=0.9cm] ->
      C [draw, circle, minimum size=0.5cm] ->
      D [draw, circle, minimum size=0.5cm];
      A -> D -> B;  % chktex 13
      A -> C;  % chktex 13
      };
      \caption{relationship between components.}
    \end{subfigure}
    \begin{subfigure}[b]{0.45\linewidth}
      \centering
      \tikz\graph[simple necklace layout, math nodes, node sep=0.5cm] {
      A' --[dashed] B' -> C' -> D';
      A' -> D' -> B';
      A' -> C';
      };
      \caption{condensed graph \(H\).}
    \end{subfigure}
    \caption{an example of graph condensation.}
    \label{fig: condensation example}  % chktex 24
  \end{figure}
\end{frame}

\begin{frame}
  \begin{theorem}\label{the: tie condensation effcient}
    Given a quasi-transitive graph \(Q\)
    and its tie component condensation \(f\),
    consider the set of all the condensations \(f_k: G_k \to T_k\),
    where \(G_k\) has the same tie structure as \(Q\)
    and \(T_k\) is a tournament:
    \(F = \{f_0, f_1, \ldots, f_{n-1}, f_n\} \).
    \(f\) is an efficient condensation in \(F\).
  \end{theorem}
\end{frame}


\subsection{property of kings}

\begin{frame}
  \begin{theorem}
    If vertex \(k\) is a king in quasi-transitive oriented graph \(G\),
    then \(k\) is adjacent to every other vertex in \(G\).
  \end{theorem}

  \begin{figure}
    \centering
    \tikz\graph[simple necklace layout, math nodes, node sep=1cm] {
    D_k [draw, circle, minimum size=2cm] ->
    k ->
    S_k [draw, circle, minimum size=2cm];
    ""; % chktex 18
    D_k --[bend left] S_k;
    };
    \caption{the rich structure of a king in a quasi-transitive oriented graph.}
    \label{fig: king in quasi-transitive}  % chktex 24
  \end{figure}
\end{frame}

\begin{frame}
  \begin{theorem}\label{the: king in quasi-transitive}
    A vertex \(k\) is a king in a quasi-transitive oriented
    graph if and only if
    \begin{itemize}
      \item \(k\) is in a trivial tie component.
      \item the result of \(k\) after tie component condensation
      is a king in the underlying tournament.
    \end{itemize}
  \end{theorem}

  \begin{figure}
    \centering
    \begin{subfigure}[b]{0.45\linewidth}
      \centering
      \tikz\graph[simple necklace layout, math nodes, node sep=0.5cm] {
      a_0; a_1;
      a_0 <- {b_0, b_1, b_2};
      a_1 <- {b_0, b_1, b_2};
      {b_0, b_1, b_2} -> c_0;
      d_0 -> {b_0, b_1, b_2};
      {a_0, a_1} -> d_0;
      c_0 -> d_0;
      b_0 -> b_1 --[dashed] b_2 --[dashed] b_0;
      a_0 --[dashed] a_1;
      };
      \caption{quasi-transitive oriented graph \(Q\).}
    \end{subfigure}
    \begin{subfigure}[b]{0.45\linewidth}
      \centering
      \tikz\graph[simple necklace layout, math nodes, node sep=0.5cm] {
      A' <- B' -> C' -> D';
      A' -> D' -> B';
      A' -> C';
      };
      \caption{underlying tournament of \(Q\).}
    \end{subfigure}
  \end{figure}
\end{frame}

\section{further problem}


% =============================================
% ENDING AND REFERENCE
% =============================================
\section{Thank You}

\begin{frame}
\end{frame}

\section{References}

\begin{frame}[allowframebreaks]
  \printbibliography[heading=bibintoc]
\end{frame}

\end{document}

